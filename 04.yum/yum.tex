
\documentclass[xcolor=dvipsnames]{beamer} % dvipsnames gives more built-in colors
\usepackage[utf8]{inputenc}
\usepackage[spanish]{babel}

\mode<presentation> {

\usetheme{CambridgeUS}

%\setbeamertemplate{footline} % To remove the footer line in all slides uncomment this line
\setbeamertemplate{footline}[page number] % To replace the footer line in all slides with a simple slide count uncomment this line

\setbeamertemplate{navigation symbols}{} % To remove the navigation symbols from

\definecolor{utfsmred}{HTML}{D60019}
\definecolor{utfsmyellow}{HTML}{F7AE00}
\definecolor{utfsmgreen}{HTML}{008452}
\definecolor{utfsmblue}{HTML}{004B85}


\newenvironment<>{rosa}[1][]{
  \setbeamercolor{block title example}{fg=white,bg=blue!75!black}%
  \begin{example}#2[#1]}{
  \end{example}
}



\usecolortheme[named=utfsmblue]{structure}
\setbeamercolor{titlelike}{parent=structure,fg=utfsmblue}
\setbeamercolor{frametitle}{fg=utfsmblue}

%\setbeamercolor{section in head/foot}{bg=Brown}
%\setbeamercolor{author in head/foot}{bg=Brown}
%\setbeamercolor{date in head/foot}{fg=Brown}

\setbeamercolor*{enumerate item}{fg=utfsmred}
\setbeamercolor*{enumerate subitem}{fg=utfsmred}
\setbeamercolor*{enumerate subsubitem}{fg=utfsmred}

\setbeamercolor*{itemize item}{fg=utfsmred}
\setbeamercolor*{itemize subitem}{fg=utfsmred}
\setbeamercolor*{itemize subsubitem}{fg=utfsmred}

\setbeamercolor{item projected}{bg=utfsmred}


\setbeamertemplate{itemize items}[square]
\setbeamertemplate{enumerate items}[default]


\setbeamercolor{section in head/foot}{bg=utfsmblue}

\setbeamercolor{block title}{bg=utfsmblue!80,fg=white}
\setbeamercolor{block title alerted}{bg=utfsmred!80,fg=white}
\setbeamercolor{block title example}{bg=utfsmgreen!80,fg=white}

\setbeamertemplate{sections/subsections in toc}[square]
\setbeamercolor{section number projected}{bg=utfsmblue,fg=white}


}

\usepackage{graphicx} % Allows including images
\usepackage{booktabs} % Allows the use of \toprule, \midrule and \bottomrule in tables

\usepackage{listings}
\lstset{ %
language=C,
basicstyle=\normalsize\ttfamily,
keywordstyle=,
numbers=none,
numberstyle=\tiny\ttfamily,
stepnumber=1,
showspaces=false,
showstringspaces=false,
showtabs=false,
breaklines=true,
frame=tb,w
framerule=0.5pt,
tabsize=4,
framexleftmargin=0.5em,
framexrightmargin=0.5em,
xleftmargin=0.5em,
xrightmargin=0.5em,
}

\title{Yum}
\subtitle{\small{\\Seminario de Desarrollo de Software - Casa Central.} }
\date{\today}
\author{\texttt{Maximiliano Osorio \\ <mosorio@inf.utfsm.cl>}}
\institute{Universidad Técnica Federico Santa María}

\hypersetup{
pdfauthor = {Maximiliano Osorio\\
 mosorio@inf.utfsm.cl},      
pdfsubject = {Informatica, },
pdfkeywords = {},  
pdfmoddate= {},          
pdfcreator = {}
}

\begin{document}

\maketitle
\begin{frame}[fragile]
    \frametitle{yum}

Yum es el Red Hat package manager permite:
\begin{itemize}
\item Consultar sobre información de los paquetes.
\item Traer paquetes de repositorios
\item Instalar o desinstalar.
\end{itemize}

\begin{block}{Dependencias}
Yum realiza resolución de dependecias de forma automatica cuando se instalar, actualiza o se remueven paquetes
\end{block}

\end{frame}


\begin{frame}[fragile]
    \frametitle{Yum}
\begin{itemize}
\item Yum puede ser configurado con nuevos repositorios o paquetes de código fuente (\texttt{package source}).
\item Además yum provee multiples plugins para extender sus capacidades.
\item Yum utiliza GPG, yum no instalará el paquete del repositorio sino está firmado con la llave correcta.
\end{itemize}


\end{frame}





\begin{frame}[fragile]
    \frametitle{Actualización}
Para verificar si existen actualizaciones
\begin{lstlisting}
yum check-update
\end{lstlisting}
\end{frame}



\begin{frame}[fragile]
    \frametitle{Actualización}
    \begin{lstlisting}[basicstyle=\small]
~]# yum check-update
Loaded plugins: langpacks, product-id, subscription-manager
Updating Red Hat repositories.
INFO:rhsm-app.repolib:repos updated: 0
PackageKit.x86_64                  0.5.8-2.el6                rhel
PackageKit-glib.x86_64             0.5.8-2.el6                rhel
PackageKit-yum.x86_64              0.5.8-2.el6                rhel
PackageKit-yum-plugin.x86_64       0.5.8-2.el6                rhel
glibc.x86_64                       2.11.90-20.el6             rhel
glibc-common.x86_64                2.10.90-22                 rhel
kernel.x86_64                      2.6.31-14.el6              rhel
rpm.x86_64                         4.7.1-5.el6                rhel
rpm-libs.x86_64                    4.7.1-5.el6                rhel
rpm-python.x86_64                  4.7.1-5.el6                rhel
yum.noarch                         3.2.24-4.el6               rhel
\end{lstlisting}
\end{frame}




\begin{frame}[fragile]
    \frametitle{Actualización}
Para verificar si existen actualizaciones
\begin{lstlisting}[basicstyle=\small]
~]# yum update rpm
Loaded plugins: langpacks, product-id, subscription-manager
Updating Red Hat repositories.
INFO:rhsm-app.repolib:repos updated: 0
Setting up Update Process
Resolving Dependencies
--> Running transaction check
---> Package rpm.x86_64 0:4.11.1-3.el7 will be updated
--> Processing Dependency: rpm = 4.11.1-3.el7 for package: rpm-libs-4.11.1-3.el7.x86_64
--> Processing Dependency: rpm = 4.11.1-3.el7 for package: rpm-python-4.11.1-3.el7.x86_64
--> Processing Dependency: rpm = 4.11.1-3.el7 for package: rpm-build-4.11.1-3.el7.x86_64
---> Package rpm.x86_64 0:4.11.2-2.el7 will be an update
--> Running transaction check
...
--> Finished Dependency Resolution

Dependencies Resolved
=============================================================================
 Package                   Arch        Version         Repository       Size
=============================================================================
Updating:
 rpm                       x86_64      4.11.2-2.el7    rhel            1.1 M
Updating for dependencies:
 rpm-build                 x86_64      4.11.2-2.el7    rhel            139 k
 rpm-build-libs            x86_64      4.11.2-2.el7    rhel             98 k
 rpm-libs                  x86_64      4.11.2-2.el7    rhel            261 k
 rpm-python                x86_64      4.11.2-2.el7    rhel             74 k

Transaction Summary
=============================================================================
Upgrade  1 Package (+4 Dependent packages)

Total size: 1.7 M
Is this ok [y/d/N]:
\end{lstlisting}
\end{frame}

\begin{frame}[fragile]
    \frametitle{Info}
En el ejemplo anterior se observar
\begin{itemize}
\item Plugins
\begin{itemize}
\item langpacks
\item product-id 
\item subscription-manager
\end{itemize}
\item  Yum por defecto es interactivo, si se utiliza \texttt{-y} automaticamente se responde que \textbf{si}
\end{itemize}
\begin{lstlisting}
yum update package_name
\end{lstlisting}
\end{frame}



\begin{frame}[fragile]
    \frametitle{Actualización}
Actualizar todo.
\begin{lstlisting}
yum update
\end{lstlisting}
Actualizar los paquetes que contengan una actualización de seguridad a la ultima versión
\begin{lstlisting}
yum update --security
\end{lstlisting}
Actualizar los paquetes que contengan una actualización de seguridad a esa versión
\begin{lstlisting}
yum update-minimal --security
\end{lstlisting}


\end{frame}


\begin{frame}[fragile]
    \frametitle{Actualización}
    \begin{itemize}
    \item kernel-4.1.6-1 está instalado
    \item kernel-4.1.6-2 ha sido lanzado como security update
    \item  kernel-4.1.6-3 ha sido lanzado como bug fix update.
    \end{itemize}

yum update-minimal --security actualiza el paquete a kernel-4.1.6-2 y  yum update --security actualiza al kernel-4.1.6-3.

\end{frame}


\begin{frame}[fragile]
    \frametitle{Buscando paquetes}
\begin{lstlisting}
yum search term
\end{lstlisting}

\begin{lstlisting}
~]$ yum search meld kompare
Loaded plugins: langpacks, langpacks, product-id, subscription-manager
Updating Red Hat repositories.
INFO:rhsm-app.repolib:repos updated: 0
============================ N/S matched: kompare =============================
kompare.x86_64 : Diff tool
...
\end{lstlisting}
\end{frame}


\begin{frame}[fragile]
    \frametitle{Filtrando paquetes}
Listar paquetes disponibles e instalados.
\begin{lstlisting}
~]$ yum list abrt-addon\* abrt-plugin\*
\end{lstlisting}

\begin{lstlisting}
~]$ yum list installed "krb?-*"
\end{lstlisting}

\begin{lstlisting}
~]$ yum search "krb?-*"
\end{lstlisting}


\end{frame}



\begin{frame}[fragile]
    \frametitle{Listando repositorios}
Para listar los repositorios: nombre,id,número de paquetes.    
\begin{lstlisting}
yum repolist
\end{lstlisting}

Para listar los repositorios: nombre,id,número de paquetes.    
\begin{lstlisting}
yum repolist -v 

...

Repo-id      : uSCI-noarch/7
Repo-name    : uSCI RHEL 7 - noarch
Repo-revision: 1430402446
Repo-updated : Thu Apr 30 09:00:46 2015
Repo-pkgs    : 21
Repo-size    : 3.3 M
Repo-baseurl : http://ftp.inf.utfsm.cl/pub/UTFSM/uSCI/RHEL/7/noarch/
Repo-expire  : 604,800 second(s) (last: Sun Nov 29 19:32:47 2015)
Repo-filename: /etc/yum.repos.d/usci.repo
\end{lstlisting}
\end{frame}

\begin{frame}[fragile]
    \frametitle{Listando repositorios}

Para listar los repositorios de forma detallada    
\begin{lstlisting}
yum repolist -v 
yum repoinfo 
\end{lstlisting}
\begin{lstlisting}
Repo-id      : uSCI-noarch/7
Repo-name    : uSCI RHEL 7 - noarch
Repo-revision: 1430402446
Repo-updated : Thu Apr 30 09:00:46 2015
Repo-pkgs    : 21
Repo-size    : 3.3 M
Repo-baseurl : http://ftp.inf.utfsm.cl/pub/UTFSM/uSCI/RHEL/7/noarch/
Repo-expire  : 604,800 second(s) (last: Sun Nov 29 19:32:47 2015)
Repo-filename: /etc/yum.repos.d/usci.repo
\end{lstlisting}
\end{frame}

\begin{frame}[fragile]
    \frametitle{Información de un paquete}
Para verificar si existen actualizaciones
\begin{lstlisting}
~]$ yum info abrt
Loaded plugins: langpacks, product-id, subscription-manager
Updating Red Hat repositories.
INFO:rhsm-app.repolib:repos updated: 0
Installed Packages
Name       : abrt
Arch       : x86_64
Version    : 1.0.7
Release    : 5.el6
Size       : 578 k
Repo       : installed
From repo  : rhel
Summary    : Automatic bug detection and reporting tool
URL        : https://fedorahosted.org/abrt/
License    : GPLv2+
Description: abrt is a tool to help users to detect defects in applications
           : and to create a bug report with all informations needed by
           : maintainer to fix it. It uses plugin system to extend its
           : functionality.
\end{lstlisting}
\end{frame}


\begin{frame}[fragile]
    \frametitle{Instalando paquetes}
Para instalar un paquete
\begin{lstlisting}
yum install package_name package_name…
yum install package_name.arch
~]# yum install sqlite.i686
~]# yum install audacious-plugins-\*

\end{lstlisting}

Si conoce el ejecutable del paquete 
\begin{lstlisting}
~]# yum install /bin/netstat    
\end{lstlisting}
\end{frame}


\begin{frame}[fragile]
    \frametitle{Buscando por ejecutable}
\begin{lstlisting}
~]# yum provides "*bin/named"
Loaded plugins: langpacks, product-id, subscription-manager
Updating Red Hat repositories.
INFO:rhsm-app.repolib:repos updated: 0
32:bind-9.7.0-4.P1.el6.x86_64 : The Berkeley Internet Name Domain (BIND)
                              : DNS (Domain Name System) server
Repo        : rhel
Matched from:
Filename    : /usr/sbin/named
\end{lstlisting}
\end{frame}

\begin{frame}[fragile]
    \frametitle{Grupos}
\begin{lstlisting}
yum group list ids kde\*
Loaded plugins: fastestmirror
There is no installed groups file.
Maybe run: yum groups mark convert (see man yum)
Loading mirror speeds from cached hostfile
 * epel: ftp.inf.utfsm.cl
Available environment groups:
   KDE Plasma Workspaces (kde-desktop-environment)
Done
\end{lstlisting}
\end{frame}

\begin{frame}[fragile]
    \frametitle{Instalar por grupos}
Multiples formas
\begin{lstlisting}
~]# yum group install "KDE Desktop"
~]# yum group install kde-desktop
~]# yum install @"KDE Desktop"
~]# yum install @kde-desktop
\end{lstlisting}

\end{frame}

\begin{frame}[fragile]
    \frametitle{Removiendo grupos}
\begin{lstlisting}
~]# yum group remove "KDE Desktop"
~]# yum group remove kde-desktop
~]# yum remove @"KDE Desktop"
~]# yum remove @kde-desktop
\end{lstlisting}
\end{frame}

\section{Yum History}

\begin{frame}[fragile]
    \frametitle{yum}

Mostrar de las ultimas 20 transacciones.
\begin{lstlisting}
yum history list
\end{lstlisting}
Todas las listas 
\begin{lstlisting}
yum history list all
\end{lstlisting}
\end{frame}


\begin{frame}[fragile]
    \frametitle{yum}
\begin{table}[]
\centering
\resizebox{\textwidth}{!}{%
\begin{tabular}{|l|l|l|}
\hline
Action & Abbreviation & Description \\ \hline
Downgrade & D & \begin{tabular}[c]{@{}l@{}}At least one package has been\\ downgraded to an older version.\end{tabular} \\ \hline
Erase & E & \begin{tabular}[c]{@{}l@{}}At least one package \\ has been removed.\end{tabular} \\ \hline
Install & I & \begin{tabular}[c]{@{}l@{}}At least one new package\\ has been installed.\end{tabular} \\ \hline
Obsoleting & O & \begin{tabular}[c]{@{}l@{}}At least one package has \\ been marked as obsolete.\end{tabular} \\ \hline
Reinstall & R & \begin{tabular}[c]{@{}l@{}}At least one package has\\ been reinstalled.\end{tabular} \\ \hline
Update & U & \begin{tabular}[c]{@{}l@{}}At least one package has\\ been updated to a newer version.\end{tabular} \\ \hline
\end{tabular}
}
\end{table}
\end{frame}


\begin{frame}[fragile]
    \frametitle{yum}
    
\begin{lstlisting}[basicstyle=\tiny]
[root@ip123 ~]# yum history summary
Loaded plugins: fastestmirror
Login user                 | Time                | Action(s)        | Altered
-------------------------------------------------------------------------------
root <root>                | Last 2 weeks        | I, U             |        4
 <mosorio>                 | Last 3 months       | I, U             |       53
root <root>                | Last 3 months       | Install          |       19
System <unset>             | Last year           | Install          |      298
root <root>                | Last year           | E, I, O, U       |      283
history summary
\end{lstlisting}
\end{frame}


\begin{frame}[fragile]
    \frametitle{yum}
\begin{lstlisting}
[root@ip123 ~]# yum history package-list openssh\*
\end{lstlisting}
\end{frame}

\begin{frame}[fragile]
    \frametitle{yum}
\begin{lstlisting}
yum history info $id
yum history info 9
\end{lstlisting}
\end{frame}

\begin{frame}[fragile]
    \frametitle{Deshacer o repetir}
\begin{lstlisting}
yum history undo id
yum history redo id

\end{lstlisting}
\end{frame}


\begin{frame}[fragile]
    \frametitle{yum.conf}
    
El archivo de configuración se encuentra \textbf{/etc/yum.conf}, esto incluye los archivos de repositorios
\begin{lstlisting}
[uSCI-noarch]
name=uSCI RHEL $releasever - noarch
failovermethod=priority
baseurl=http://ftp.inf.utfsm.cl/pub/UTFSM/uSCI/RHEL/$releasever/noarch/
enabled=1
metadata_expire=7d
gpgcheck=1
gpgkey=file:///etc/pki/rpm-gpg/RPM-GPG-KEY-uSCI
\end{lstlisting}
\end{frame}

\begin{frame}[fragile]
    \frametitle{yum.conf}
    La configuración se puede ver y modificar con \textbf{yum-config-manager}
\begin{lstlisting}
~]$ yum-config-manager main \*

\end{lstlisting}
\end{frame}

\begin{frame}[fragile]
    \frametitle{Añadiendo, activando o desactivando repositorios}
\begin{lstlisting}
$yum-config-manager --add-repo repository_url
\end{lstlisting}

\begin{lstlisting}
~]# yum-config-manager --add-repo http://www.example.com/example.repo
Loaded plugins: langpacks, product-id, subscription-manager
adding repo from: http://www.example.com/example.repo
grabbing file http://www.example.com/example.repo to /etc/yum.repos.d/example.repo
example.repo                                             |  413 B     00:00
repo saved to /etc/yum.repos.d/example.repo
\end{lstlisting}

\end{frame}

\begin{frame}[fragile]
    \frametitle{Activando repos}
\begin{lstlisting}
~]# yum-config-manager --enable example\*
Loaded plugins: langpacks, product-id, subscription-manager
============================== repo: example ==============================
[example]
bandwidth = 0
base_persistdir = /var/lib/yum/repos/x86_64/7Server
baseurl = http://www.example.com/repo/7Server/x86_64/
cache = 0
cachedir = /var/cache/yum/x86_64/7Server/example
[output truncated]
\end{lstlisting}
\end{frame}

\begin{frame}[fragile]
    \frametitle{Activando todos los repos}
\begin{lstlisting}
~]# yum-config-manager --enable \*
Loaded plugins: langpacks, product-id, subscription-manager
============================== repo: example ==============================
[example]
bandwidth = 0
base_persistdir = /var/lib/yum/repos/x86_64/7Server
baseurl = http://www.example.com/repo/7Server/x86_64/
cache = 0
cachedir = /var/cache/yum/x86_64/7Server/example
[output truncated]
\end{lstlisting}
\end{frame}


\begin{frame}[fragile]
    \frametitle{Creando un repositorio}
    
\begin{lstlisting}
yum install createrepo
    reposync -l --repoid=$repo --downloadcomps --download-metadata --download_path=$path
    createrepo -v $path/$repo
\end{lstlisting}

\end{frame}

\section{Cron}

\begin{frame}[fragile]
    \frametitle{yum}

Cron es un daemon que puede hacer un scheduler de  tareas en cierto puntos de tiempos: dia del mes, semana.
\begin{lstlisting}
~]# yum install cronie cronie-anacron
systemctl start crond.service
systemctl enable crond.service

\end{lstlisting}
\end{frame}

\begin{frame}[fragile]
    \frametitle{crontab}

El archivo de configuración de cron jobs se encuentra en \textbf{/etc/crontab}
\begin{lstlisting}[basicstyle=\tiny]
SHELL=/bin/bash
PATH=/sbin:/bin:/usr/sbin:/usr/bin
MAILTO=root
HOME=/
# For details see man 4 crontabs
# Example of job definition:
# .---------------- minute (0 - 59)
# | .------------- hour (0 - 23)
# | | .---------- day of month (1 - 31)
# | | | .------- month (1 - 12) OR jan,feb,mar,apr ...
# | | | | .---- day of week (0 - 6) (Sunday=0 or 7) OR sun,mon,tue,wed,thu,fri,sat
# | | | | |
# * * * * * user-name command to be executed
\end{lstlisting}
\end{frame}

\begin{frame}[fragile]
    \frametitle{crontab}

\begin{lstlisting}[basicstyle=\small]
00 11,16 * * * /home/ramesh/bin/incremental-backup
00 – 0th Minute (Top of the hour)
11,16 – 11 AM and 4 PM
* – Every day
* – Every month
* – Every day of the week
\end{lstlisting}

\begin{lstlisting}
00 09-18 * * * /home/ramesh/bin/check-db-status
00 – 0th Minute (Top of the hour)
09-18 – 9 am, 10 am,11 am, 12 am, 1 pm, 2 pm, 3 pm, 4 pm, 5 pm, 6 pm
* – Every day
* – Every month
* – Every day of the week
\end{lstlisting}

\end{frame}


\end{document}


