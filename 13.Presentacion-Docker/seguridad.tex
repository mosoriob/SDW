
\documentclass[xcolor=dvipsnames]{beamer} % dvipsnames gives more built-in colors
\usepackage[utf8]{inputenc}
\usepackage[spanish]{babel}

\mode<presentation> {

\usetheme{CambridgeUS}

%\setbeamertemplate{footline} % To remove the footer line in all slides uncomment this line
\setbeamertemplate{footline}[page number] % To replace the footer line in all slides with a simple slide count uncomment this line

\setbeamertemplate{navigation symbols}{} % To remove the navigation symbols from

\definecolor{utfsmred}{HTML}{D60019}
\definecolor{utfsmyellow}{HTML}{F7AE00}
\definecolor{utfsmgreen}{HTML}{008452}
\definecolor{utfsmblue}{HTML}{004B85}


\newenvironment<>{rosa}[1][]{
  \setbeamercolor{block title example}{fg=white,bg=blue!75!black}%
  \begin{example}#2[#1]}{
  \end{example}
}

\usepackage{hyperref}
\definecolor{links}{HTML}{2A1B81}
\hypersetup{colorlinks,linkcolor=,urlcolor=links}

\usecolortheme[named=utfsmblue]{structure}
\setbeamercolor{titlelike}{parent=structure,fg=utfsmblue}
\setbeamercolor{frametitle}{fg=utfsmblue}

%\setbeamercolor{section in head/foot}{bg=Brown}
%\setbeamercolor{author in head/foot}{bg=Brown}
%\setbeamercolor{date in head/foot}{fg=Brown}

\setbeamercolor*{enumerate item}{fg=utfsmred}
\setbeamercolor*{enumerate subitem}{fg=utfsmred}
\setbeamercolor*{enumerate subsubitem}{fg=utfsmred}

\setbeamercolor*{itemize item}{fg=utfsmred}
\setbeamercolor*{itemize subitem}{fg=utfsmred}
\setbeamercolor*{itemize subsubitem}{fg=utfsmred}

\setbeamercolor{item projected}{bg=utfsmred}


\setbeamertemplate{itemize items}[square]
\setbeamertemplate{enumerate items}[default]


\setbeamercolor{section in head/foot}{bg=utfsmblue}

\setbeamercolor{block title}{bg=utfsmblue!80,fg=white}
\setbeamercolor{block title alerted}{bg=utfsmred!80,fg=white}
\setbeamercolor{block title example}{bg=utfsmgreen!80,fg=white}

\setbeamertemplate{sections/subsections in toc}[square]
\setbeamercolor{section number projected}{bg=utfsmblue,fg=white}


}

\usepackage[normalem]{ulem}

\usepackage{graphicx} % Allows including images
\usepackage{booktabs} % Allows the use of \toprule, \midrule and \bottomrule in tables
\usepackage{listings}
\lstset{ %
language=C,
basicstyle=\tiny\ttfamily,
keywordstyle=,
numbers=none,
numberstyle=\tiny\ttfamily,
stepnumber=1,
showspaces=false,
showstringspaces=false,
showtabs=false,
breaklines=true,
frame=tb,
framerule=0.5pt,
tabsize=4,
framexleftmargin=0.5em,
framexrightmargin=0.5em,
xleftmargin=0.5em,
xrightmargin=0.5em,
}

%----------------------------------------------------------------------------------------
%	TITLE PAGE
%----------------------------------------------------------------------------------------

\title{MySQL Replication}
\subtitle{\small{Seminario de Desarrollo de Software - Casa Central.}}
\author{Maximiliano Osorio\\\small{mosorio@inf.utfsm.cl}} 
\institute[UTFSM]
{
Universidad Técnica Federico Santa María
\medskip
}
\date{\today} % Date, can be changed to a custom date

\begin{document}

%-=-=-=-=-=-=-=-=-=-=-=-=-=-=-=-=-=-=-=-=-=-=-=-=
%
%	TITLE PAGE
%
%-=-=-=-=-=-=-=-=-=-=-=-=-=-=-=-=-=-=-=-=-=-=-=-=

\maketitle




\begin{frame}{Docker containers}
	\begin{figure}
		\centering
\includegraphicscopyright[width=\textwidth,height=0.8\textheight,keepaspectratio]{images/shipping-container-for-code}{Imagen: docker.com}
	\end{figure}
\end{frame}


%-=-=-=-=-=-=-=-=-=-=-=-=-=-=-=-=-=-=-=-=-=-=-=-=
%
%
%-=-=-=-=-=-=-=-=-=-=-=-=-=-=-=-=-=-=-=-=-=-=-=-=
\section*{Docker}

\begin{frame}{Docker}

	\begin{itemize}
		\item Docker is an open platform for developing, shipping, and running applications.
		\item Docker allows you to package an application with all of its dependencies into a standardized unit for software development.
	\end{itemize}

\end{frame}

\begin{frame}{Docker Benefits}
	\begin{itemize}
\item Fast (deployment, migration, restarts)
\item Secure
\item Lightweight (save disk \& CPU)
\item Open Source
\item Portable software
\item Microservices and integrations (APIs)
\item Simplify DevOps
\item Version control capabilities
	\end{itemize}
\end{frame}


\begin{frame}{}
	\begin{figure}
		\centering
		\includegraphicscopyright[width=\linewidth]{images/container_vs_vm.jpg}{\href{http://www.qafe.com/what-is-docker-why-en-how-use-it/}{qafe.com}}
	\end{figure}
\end{frame}




\begin{frame}[fragile]
\frametitle{Namespaces}

\begin{itemize}
	\item Provide processes with their own view of the system.
	\item Multiple namespaces: pid, net, mnt, uts, ipc, user
\end{itemize}
Docker: Namespaces provide the first and most straightforward form of isolation: processes running within a container cannot see, and even less affect, processes running in another container, or in the host system.
   
\end{frame}

\begin{frame}{Docker registry}

\begin{figure}[H]
  \centering
  \includegraphicscopyright[width=\textwidth,height=0.6\textheight,keepaspectratio]{images/process.png}{http://cloud.51cto.com/}
    \label{fig:dynamic}
\end{figure}	
\end{frame}

%\begin{frame}[fragile]
%\frametitle{}
%\begin{itemize}
%	\item Namespaces: PID, Mount, Network, IPC.
%	\item Cgroups: Limit CPU, Memory, IO.
%	\item Capabilities: reduced root access.
%	\item Special kernel modules like AppArmor, SElinux - Provides granular control over Kernel resources.
%\end{itemize}
%\end{frame}


\subsection{Images}

\section{Containers don't contain}
\section*{Tips}

\begin{frame}[fragile]
\frametitle{Why don't containers contain?}
Dan Walsh \footnote{Mr. SElinux} meant that not all resources that a container has access to are namespaces.
\begin{itemize}
	\item UID (by default) If a user is root inside a container and breaks out of the container, that user will be root on the host. 
	\item The kernel keyring
	\item The kernel itself and any kernel modules
	\item The system time \texttt{SYS\_TIME}
	\item Namespaces have vulnerabilities!
\end{itemize}
\end{frame}

\subsection{Containers and namespacing}

\subsection{UID/GID}

\begin{frame}[fragile]
\frametitle{uid/gid 1}
\begin{itemize}
	\item  The linux kernel is responsible for managing the uid and gid space
	\item the uid and gid that created the process are examined by the kernel to determine if it has enough privileges to modify the file
\end{itemize}

The username isn't used here, the uid is used!
\end{frame}

\begin{frame}[fragile]
\frametitle{}
\begin{lstlisting}
[mosorio@host ~]$  docker run -d ubuntu:latest sleep infinity
\end{lstlisting}

\begin{lstlisting}
root     30636  0.0  0.0   4372   364 ?        Ss   20:43   0:00          \_ sleep infinity	
\end{lstlisting}
\end{frame}

\begin{frame}[fragile]
\frametitle{uid/gid 2}
Does root inside the container == root outside the container?
\begin{itemize}
	\item  Yes, because, as I mentioned, there’s a single kernel and a single, shared pool of uids and gids.
\end{itemize}
\end{frame}


\begin{frame}[fragile]
\frametitle{uid/gid 3}
\begin{lstlisting}
[mosorio@host ~]$ docker run -it -u $(id -u $USER):$(id -g $GROUP) centos:7 bash
bash-4.2$ id
uid=1002 gid=1002 groups=1002
\end{lstlisting}
\end{frame}

\begin{frame}[fragile]
\frametitle{User Namespaces}
As of Docker 1.10, you can enable user namespacing by starting the
kernel with the \texttt{--userns-remap}.
\begin{itemize}
	\item WARNING: Using Docker volumes becomes more complex as the changed UIDs affect access privileges.
\end{itemize}
\end{frame}

\begin{frame}[fragile]
\frametitle{}
\begin{lstlisting}
root@container:/# id
uid=0(root) gid=0(root) groups=0(root)
\end{lstlisting}

\begin{lstlisting}
root@host:/# ps -eaf | grep bash
231072 4080 4040 ... bash
\end{lstlisting}

userId 0 in container is mapped to 231072 in host
\begin{lstlisting}
root@host:/# cat /proc/4080/uid_map
0 231072 65536
\end{lstlisting}
\end{frame}

\begin{frame}[fragile]
\frametitle{Docker daemon}
Any user who can start and run Docker containers effectively has root access to the host. For example, consider that you can run the following:
\begin{lstlisting}
usermod -aG docker dummyuser
\end{lstlisting}
\end{frame}


\begin{frame}[fragile]
\frametitle{Docker daemon}
\begin{lstlisting}
[mosorio@host ~]$ docker run -v /:/homeroot -it debian bash
root@b33b601c262c:/# ls /homeroot/root/.ssh
authorized_keys  config  id_rsa  known_hosts
\end{lstlisting}
\end{frame}


\subsection{Remove setuid/setgid Binaries}

\begin{frame}[fragile]
\frametitle{Remove setuid/setgid Binaries}
Chances are that your application doesn't need any setuid or
setgid binaries.
\begin{lstlisting}
 find / -perm +6000 -type f -exec chmod a-s {} \;
\end{lstlisting}
\end{frame}

\subsection{DDoS - cgroups}
\begin{frame}[fragile]
\frametitle{Limit Memory and CPU}
Limiting memory protects against both DoS attacks and applications with memory leaks 
\begin{lstlisting}
docker run -m 128m --memory-swap 128m ubuntu:latest
docker run -d --name load1 -c 2048
docker run -d --restart=on-failure:10 my-flaky-image
\end{lstlisting}
Also, apply resource limits (ulimits)!
\end{frame}


\subsection{Capabilities}

\begin{frame}[fragile]
\frametitle{Capabilities}
The Linux kernel defines sets of privileges—called capabilities—that can be assigned to processes to provide them with greater access to the system.

\begin{lstlisting}
$ docker run --cap-drop all debian chown 100 /tmp
chown: changing ownership of '/tmp': Operation not permitted
\end{lstlisting}
\end{frame}

\subsection{Images}


\begin{frame}{}
	\begin{figure}
		\centering
		\includegraphicscopyright[width=\textwidth,height=0.9\textheight,keepaspectratio]{images/dockerhub-fig-3}{\href{http://dl.acm.org/citation.cfm?id=3029832}{A Study of Security Vulnerabilities on Docker Hub}}

	\end{figure}
\end{frame}


\begin{frame}{}
	\begin{figure}
		\centering
		\includegraphicscopyright[width=\textwidth,height=0.9\textheight,keepaspectratio]{images/dockerhub-fig-5}{\href{http://dl.acm.org/citation.cfm?id=3029832}{A Study of Security Vulnerabilities on Docker Hub}}

	\end{figure}
\end{frame}
\section{Docker is about running random crap from the internet as root on your host.}
\begin{frame}{}
	\begin{figure}
		\centering
		\includegraphicscopyright[width=\textwidth,height=0.9\textheight,keepaspectratio]{images/devopssec}{\href{http://devops.tumblr.com}{http://devops.tumblr.com}}

	\end{figure}
\end{frame}
\begin{frame}[fragile]
\frametitle{Applying Updates}
The ability to quickly apply updates to a running system is critical to maintaining security, especially when vulnerabilities are disclosed in common utilities and frameworks.
\begin{itemize}
	\item   Identify images that require updating
	\item Get or create an updated version of each base image
	\item For each dependent image, run docker build with the \-\-nocache argument. Again, push these images
	\item Restart
	\item Clean
\end{itemize}
\end{frame}

\section{SElinux}

\begin{frame}[fragile]
\frametitle{SElinux}
The SELinux, or Security Enhanced Linux, module was developed by the United States National Security Agency (NSA) as an implementation of what they call mandatory access control (MAC)
\end{frame}
 

\begin{frame}{}
	\begin{figure}
		\centering
		\includegraphicscopyright[width=\textwidth,height=0.9\textheight,keepaspectratio]{images/dogs-and-cats}{\href{https://people.redhat.com/duffy/selinux/selinux-coloring-book_A4-Stapled.pdf}{RedHat}}
	\end{figure}
\end{frame}

\begin{frame}{}
	\begin{figure}
		\centering
		\includegraphicscopyright[width=\textwidth,height=0.2\textheight,keepaspectratio]{selinux/selinux2}{\href{https://people.redhat.com/duffy/selinux/selinux-coloring-book_A4-Stapled.pdf}{RedHat}}
		\includegraphicscopyright[width=\textwidth,height=0.9\textheight,keepaspectratio]{selinux/selinux3}{\href{https://people.redhat.com/duffy/selinux/selinux-coloring-book_A4-Stapled.pdf}{RedHat}}
	\end{figure}
\end{frame}

\begin{frame}{}
	\begin{figure}
		\centering
		\includegraphicscopyright[width=\textwidth,height=0.9\textheight,keepaspectratio]{selinux/selinux4}{\href{https://people.redhat.com/duffy/selinux/selinux-coloring-book_A4-Stapled.pdf}{RedHat}}

	\end{figure}
\end{frame}



\begin{frame}[fragile]
\frametitle{Recommendations}
\begin{itemize}
	\item   Only run container images from trusted parties.
Container applications should drop privileges or run without privileges whenever possible.
	\item Make sure the kernel is always updated with the latest security fixes; the security kernel is critical.

	\item Do not disable security features of the host operating system.
	\item Examine your container images for security flaws and make sure the provider fixes them in a timely manner.
		\item Further security can be provided at the kernel level by running hardened kernels and using security modules such as AppArmor or SELinux.
\end{itemize}
\end{frame}
\nocite{*}


\bibliographystyle{ieeetr}
      \bibliography{ref.bib}
\end{document}
