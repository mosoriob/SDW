
\documentclass[xcolor=dvipsnames]{beamer} % dvipsnames gives more built-in colors
\usepackage[utf8]{inputenc}
\usepackage[spanish]{babel}

\mode<presentation> {

\usetheme{CambridgeUS}

%\setbeamertemplate{footline} % To remove the footer line in all slides uncomment this line
\setbeamertemplate{footline}[page number] % To replace the footer line in all slides with a simple slide count uncomment this line

\setbeamertemplate{navigation symbols}{} % To remove the navigation symbols from

\definecolor{utfsmred}{HTML}{D60019}
\definecolor{utfsmyellow}{HTML}{F7AE00}
\definecolor{utfsmgreen}{HTML}{008452}
\definecolor{utfsmblue}{HTML}{004B85}


\newenvironment<>{rosa}[1][]{
  \setbeamercolor{block title example}{fg=white,bg=blue!75!black}%
  \begin{example}#2[#1]}{
  \end{example}
}

\usepackage{hyperref}
\definecolor{links}{HTML}{2A1B81}
\hypersetup{colorlinks,linkcolor=,urlcolor=links}

\usecolortheme[named=utfsmblue]{structure}
\setbeamercolor{titlelike}{parent=structure,fg=utfsmblue}
\setbeamercolor{frametitle}{fg=utfsmblue}

%\setbeamercolor{section in head/foot}{bg=Brown}
%\setbeamercolor{author in head/foot}{bg=Brown}
%\setbeamercolor{date in head/foot}{fg=Brown}

\setbeamercolor*{enumerate item}{fg=utfsmred}
\setbeamercolor*{enumerate subitem}{fg=utfsmred}
\setbeamercolor*{enumerate subsubitem}{fg=utfsmred}

\setbeamercolor*{itemize item}{fg=utfsmred}
\setbeamercolor*{itemize subitem}{fg=utfsmred}
\setbeamercolor*{itemize subsubitem}{fg=utfsmred}

\setbeamercolor{item projected}{bg=utfsmred}


\setbeamertemplate{itemize items}[square]
\setbeamertemplate{enumerate items}[default]


\setbeamercolor{section in head/foot}{bg=utfsmblue}

\setbeamercolor{block title}{bg=utfsmblue!80,fg=white}
\setbeamercolor{block title alerted}{bg=utfsmred!80,fg=white}
\setbeamercolor{block title example}{bg=utfsmgreen!80,fg=white}

\setbeamertemplate{sections/subsections in toc}[square]
\setbeamercolor{section number projected}{bg=utfsmblue,fg=white}


}
\usepackage{lmodern}  % for bold teletype font
\usepackage{amsmath}  % for \hookrightarrow
\usepackage{xcolor}   % for \textcolor
\usepackage{listings}
\usepackage[normalem]{ulem}

\usepackage{graphicx} % Allows including images
\usepackage{booktabs} % Allows the use of \toprule, \midrule and \bottomrule in tables
\usepackage{listings}
\lstset{
  basicstyle=\ttfamily,
  columns=fullflexible,
  frame=single,
  breaklines=true,
  postbreak=\mbox{\textcolor{red}{$\hookrightarrow$}\space},
}


%----------------------------------------------------------------------------------------
%	TITLE PAGE
%----------------------------------------------------------------------------------------

\title{Nginx - location block}
\subtitle{\small{Seminario de Desarrollo de Software - Casa Central.}}
\author{Maximiliano Osorio\\\small{mosorio@inf.utfsm.cl}} 
\institute[UTFSM]
{
Universidad Técnica Federico Santa María
\medskip
}
\date{\today} % Date, can be changed to a custom date

\begin{document}
	
%-=-=-=-=-=-=-=-=-=-=-=-=-=-=-=-=-=-=-=-=-=-=-=-=
%
%	TITLE PAGE
%
%-=-=-=-=-=-=-=-=-=-=-=-=-=-=-=-=-=-=-=-=-=-=-=-=





%\begin{block}{Dependencias}
%\end{block}
\maketitle
\section{Location block}

\begin{frame}[fragile]
\frametitle{Location block}
De forma similar al serverblock, Nginx tiene un algoritmo para decidir distintos comportamiento por URI
\begin{block}{Location blocks}
Son usados para decidir como procesa un request URI (la parte del request que viene después del dominio o IP)
\end{block}

\end{frame}
%------------------------------------------------


\begin{frame}[fragile]
\frametitle{Location block syntax}
Generalmente son de la siguiente forma
\begin{lstlisting}
location optional_modifier location_match {

    . . .

}

\end{lstlisting}
\end{frame}
%------------------------------------------------


\begin{frame}[fragile]
\frametitle{Tipos}
\begin{itemize}
  \item (none)
  \item \verb|=|
  \item \verb|~|
  \item \verb|~*|
  \item \verb|^~|
\end{itemize}

\end{frame}
%------------------------------------------------


\begin{frame}[fragile]
\frametitle{None}

Si no hay modificadores presentes, el location es interpretado como un prefix match. \\
Esto significa que la ubicación dada se comparará con el inicio de la URI de la solicitud para determinar una coincidencia.

\end{frame}
%------------------------------------------------


\begin{frame}[fragile]
\frametitle{None}
/site, /site/page1/index.html, or /site/index.html:
\begin{lstlisting}
location /site {

    . . .

}
\end{lstlisting}
\end{frame}
%------------------------------------------------


\begin{frame}[fragile]
\frametitle{Equal}
Para realizar un match exacto. \\
No va a responder a /page1/index.html

\begin{lstlisting}
location = /page1 {

    . . .

}
\end{lstlisting}
\end{frame}
%------------------------------------------------


\begin{frame}[fragile]
\frametitle{Expresión regular case sensitive}
Una expresión regular case-sensitive\\

Funciona para /tortoise.jpg, pero no para /FLOWER.PNG:

\begin{lstlisting}
location ~ \.(jpe?g|png|gif|ico)$ {

    . . .

}
\end{lstlisting}
\end{frame}
%------------------------------------------------


\begin{frame}[fragile]
\frametitle{Expresión regular case insensitive}}
Una expresión regular case-insensitive\\

\begin{lstlisting}
location ~* \.(jpe?g|png|gif|ico)$ {

    . . .

}
\end{lstlisting}
\end{frame}
%------------------------------------------------

\begin{frame}[fragile]
\frametitle{Best non regex match}
Cuando este modificador es usado, la url que hace match usará esta configuración sin importa si existe una expresión regular.0

/images/lala.gif va a configuración D
\begin{lstlisting}
location ~* \.(gif|jpg|jpeg)$ {
    [ configuration E ]
}
                    

location ^~ /images/ {
    [ configuration D ]
}
\end{lstlisting}
\end{frame}
%------------------------------------------------



\begin{frame}[fragile]
\frametitle{Tarea}
Entender el orden

https://github.com/detailyang/nginx-location-match-visible
\end{frame}
%------------------------------------------------

\begin{frame}[fragile]
\frametitle{php-fpm + Nginx}
\begin{lstlisting}

location ~ \.php$ {
        try_files $uri =404;
        fastcgi_pass 127.0.0.1:9000;
        #fastcgi_pass unix:/var/run/taxis.sock;
        fastcgi_index index.php;
        include fastcgi_params;
}
\end{lstlisting}
\end{frame}
%------------------------------------------------


%------------------------------------------------

\end{document}


