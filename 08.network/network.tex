
\documentclass[xcolor=dvipsnames]{beamer} % dvipsnames gives more built-in colors
\usepackage[utf8]{inputenc}
\usepackage[spanish]{babel}

\mode<presentation> {

\usetheme{CambridgeUS}

%\setbeamertemplate{footline} % To remove the footer line in all slides uncomment this line
\setbeamertemplate{footline}[page number] % To replace the footer line in all slides with a simple slide count uncomment this line

\setbeamertemplate{navigation symbols}{} % To remove the navigation symbols from

\definecolor{utfsmred}{HTML}{D60019}
\definecolor{utfsmyellow}{HTML}{F7AE00}
\definecolor{utfsmgreen}{HTML}{008452}
\definecolor{utfsmblue}{HTML}{004B85}


\newenvironment<>{rosa}[1][]{
  \setbeamercolor{block title example}{fg=white,bg=blue!75!black}%
  \begin{example}#2[#1]}{
  \end{example}
}

\usepackage{hyperref}
\definecolor{links}{HTML}{2A1B81}
\hypersetup{colorlinks,linkcolor=,urlcolor=links}

\usecolortheme[named=utfsmblue]{structure}
\setbeamercolor{titlelike}{parent=structure,fg=utfsmblue}
\setbeamercolor{frametitle}{fg=utfsmblue}

%\setbeamercolor{section in head/foot}{bg=Brown}
%\setbeamercolor{author in head/foot}{bg=Brown}
%\setbeamercolor{date in head/foot}{fg=Brown}

\setbeamercolor*{enumerate item}{fg=utfsmred}
\setbeamercolor*{enumerate subitem}{fg=utfsmred}
\setbeamercolor*{enumerate subsubitem}{fg=utfsmred}

\setbeamercolor*{itemize item}{fg=utfsmred}
\setbeamercolor*{itemize subitem}{fg=utfsmred}
\setbeamercolor*{itemize subsubitem}{fg=utfsmred}

\setbeamercolor{item projected}{bg=utfsmred}


\setbeamertemplate{itemize items}[square]
\setbeamertemplate{enumerate items}[default]


\setbeamercolor{section in head/foot}{bg=utfsmblue}

\setbeamercolor{block title}{bg=utfsmblue!80,fg=white}
\setbeamercolor{block title alerted}{bg=utfsmred!80,fg=white}
\setbeamercolor{block title example}{bg=utfsmgreen!80,fg=white}

\setbeamertemplate{sections/subsections in toc}[square]
\setbeamercolor{section number projected}{bg=utfsmblue,fg=white}


}

\usepackage[normalem]{ulem}

\usepackage{graphicx} % Allows including images
\usepackage{booktabs} % Allows the use of \toprule, \midrule and \bottomrule in tables
\usepackage{listings}
\lstset{ %
language=C,
basicstyle=\normalsize\ttfamily,
keywordstyle=,
numbers=none,
numberstyle=\tiny\ttfamily,
stepnumber=1,
showspaces=false,
showstringspaces=false,
showtabs=false,
breaklines=true,
frame=tb,
framerule=0.5pt,
tabsize=4,
framexleftmargin=0.5em,
framexrightmargin=0.5em,
xleftmargin=0.5em,
xrightmargin=0.5em,
}

%----------------------------------------------------------------------------------------
%	TITLE PAGE
%----------------------------------------------------------------------------------------

\title{Red}
\subtitle{\small{Seminario de Desarrollo de Software - Casa Central.}}
\author{Maximiliano Osorio\\\small{mosorio@inf.utfsm.cl}} 
\institute[UTFSM]
{
Universidad Técnica Federico Santa María
\medskip
}
\date{\today} % Date, can be changed to a custom date

\begin{document}
	
%-=-=-=-=-=-=-=-=-=-=-=-=-=-=-=-=-=-=-=-=-=-=-=-=
%
%	TITLE PAGE
%
%-=-=-=-=-=-=-=-=-=-=-=-=-=-=-=-=-=-=-=-=-=-=-=-=





%\begin{block}{Dependencias}
%\end{block}
\maketitle
\section{Materia}

\begin{frame}[fragile]
\frametitle{Archivos importantes}
\begin{itemize}
	\item \texttt{/etc/hosts}
	\item \texttt{/etc/resolv.conf}
	\item \texttt{/etc/sysconfig/network}
	\item \texttt{/etc/sysconfig/network-scripts/ifcfg-interface-name}
\end{itemize}
\end{frame}
%------------------------------------------------


\begin{frame}[fragile]
\frametitle{\texttt{/etc/hosts}}

El propósito principal es resolver los nombres que no pueden ser resueltos

\begin{lstlisting}
10.10.15.200   grupo01.mosorio.me grupo01
\end{lstlisting}
\end{frame}
%------------------------------------------------


\begin{frame}[fragile]
\frametitle{\texttt{/etc/resolv.conf}}

El archivo especifica dirección IP de los servidores DNS y el search domain

\begin{lstlisting}
 resolv.conf(5)
\end{lstlisting}


\end{frame}
%------------------------------------------------

\begin{frame}[fragile]
\frametitle{\texttt{/etc/sysconfig/network}}



Este archivo especifica el ruteo y la información del host para todas las interfaces.

Para mayor información revisar la \href{https://access.redhat.com/documentation/en-US/Red_Hat_Enterprise_Linux/6/html/Deployment_Guide/ch-Network_Interfaces.html#s1-networkscripts-files}{documentación de RedHat}
\end{frame}
%------------------------------------------------


\begin{frame}[fragile]
\frametitle{\texttt{ifcfg-interface-name}}

\texttt{/etc/sysconfig/network-scripts/ifcfg-interface-name}. \\ Para cada interfaz de red, existe un archivo de configuración
\end{frame}
%------------------------------------------------

\begin{frame}[fragile]
\frametitle{IP estática}
\begin{lstlisting}
DEVICE=eth0
BOOTPROTO=none
ONBOOT=yes
NETMASK=255.255.255.0
IPADDR=10.0.1.27
DNS1=200.1.19.129
DNS2=204.87.169.22
\end{lstlisting}
\end{frame}
%------------------------------------------------

\begin{frame}[fragile]
\frametitle{DHCP}
\begin{lstlisting}
DEVICE=eth0
BOOTPROTO=dhcp
ONBOOT=yes
\end{lstlisting}
\end{frame}
%---------------


\section{Curso}

\begin{frame}[fragile]
\frametitle{Laboratorio}
\pause
\begin{lstlisting}
2: eth0: <BROADCAST,MULTICAST,UP,LOWER_UP> mtu 1500 qdisc pfifo_fast state UP qlen 1000
    link/ether 00:1a:4a:78:92:5d brd ff:ff:ff:ff:ff:ff
    inet 10.10.15.200/24 brd 10.10.15.255 scope global dynamic eth0
       valid_lft 2409sec preferred_lft 2409sec
    inet6 fe80::21a:4aff:fe78:925d/64 scope link
       valid_lft forever preferred_lft forever
3: eth1: <BROADCAST,MULTICAST,UP,LOWER_UP> mtu 1500 qdisc pfifo_fast state UP qlen 1000
    link/ether 00:1a:4a:78:92:29 brd ff:ff:ff:ff:ff:ff
    inet 10.10.21.34/24 brd 10.10.21.255 scope global dynamic eth1
       valid_lft 3599sec preferred_lft 3599sec
    inet6 fe80::be0e:768b:5e79:3d1f/64 scope link
       valid_lft forever preferred_lft forever
\end{lstlisting}
\end{frame}

\begin{frame}[fragile]
\frametitle{Pregunta}

\begin{lstlisting}
::1
::
\end{lstlisting}
\end{frame}
%------------------------------------------------

%------------------------------------------------

\begin{frame}[fragile]
\frametitle{hostnamectl}
\begin{lstlisting}
hostnamectl set-hostname grupoXX-m1
\end{lstlisting}
\end{frame}


%------------------------------------------------

\begin{frame}[fragile]
\frametitle{eth0}
\begin{lstlisting}
DEVICE="eth0"
ONBOOT=yes
BOOTPROTO=none
IPADDR=${IP_PRIVADA}
NETMASK=255.255.255.0
ONBOOT=yes
IPV6INIT=no
\end{lstlisting}
\end{frame}
%------------------------------------------------

\begin{frame}[fragile]
\frametitle{eth1}
\begin{lstlisting}
DEVICE="eth1"
ONBOOT=yes
BOOTPROTO=none
IPADDR=${IP_PUBLICA}
NETMASK=255.255.255.248
GATEWAY=204.87.169.65
DNS1=200.1.19.129
DNS2=204.87.169.22
IPV6INIT=no

\end{lstlisting}
\end{frame}
%------------------------------------------------


\begin{frame}[fragile]
\frametitle{Daemons}

Desde RedHat 7, el servicio utilizado por defecto es NetworkManager. \sout{Pero durante en este laboratorio se utilizará network}. Por lo tanto,

\begin{lstlisting}
systemctl start NetworkManager
systemctl enable NetworkManager
\end{lstlisting}
\end{frame}
%------------------------------------------------

\begin{frame}[fragile]
\frametitle{nm-cli}
Configuración de la ip pública
\begin{lstlisting}
nmcli con add con-name eth1 type ethernet ifname eth1 ip4 204.87.169.127/29 gw4 204.87.169.65
nmcli connection up eth1
\end{lstlisting}
\begin{lstlisting}
nmcli connection modify eth1 ipv4.dns "200.1.19.16 8.8.8.8"
nmcli connection up eth1
\end{lstlisting}
\end{frame}
%------------------------------------------------

\begin{frame}[fragile]
\frametitle{nm-cli}
Configuración del DNS
\begin{lstlisting}
nmcli connection modify eth1 ipv4.dns "200.1.19.16 8.8.8.8"
nmcli connection up eth1
\end{lstlisting}
\end{frame}
%------------------------------------------------



\end{document}


