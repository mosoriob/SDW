
\documentclass[xcolor=dvipsnames]{beamer} % dvipsnames gives more built-in colors
\usepackage[utf8]{inputenc}
\usepackage[spanish]{babel}

\mode<presentation> {

\usetheme{CambridgeUS}

%\setbeamertemplate{footline} % To remove the footer line in all slides uncomment this line
\setbeamertemplate{footline}[page number] % To replace the footer line in all slides with a simple slide count uncomment this line

\setbeamertemplate{navigation symbols}{} % To remove the navigation symbols from

\definecolor{utfsmred}{HTML}{D60019}
\definecolor{utfsmyellow}{HTML}{F7AE00}
\definecolor{utfsmgreen}{HTML}{008452}
\definecolor{utfsmblue}{HTML}{004B85}


\newenvironment<>{rosa}[1][]{
  \setbeamercolor{block title example}{fg=white,bg=blue!75!black}%
  \begin{example}#2[#1]}{
  \end{example}
}



\usecolortheme[named=utfsmblue]{structure}
\setbeamercolor{titlelike}{parent=structure,fg=utfsmblue}
\setbeamercolor{frametitle}{fg=utfsmblue}

%\setbeamercolor{section in head/foot}{bg=Brown}
%\setbeamercolor{author in head/foot}{bg=Brown}
%\setbeamercolor{date in head/foot}{fg=Brown}

\setbeamercolor*{enumerate item}{fg=utfsmred}
\setbeamercolor*{enumerate subitem}{fg=utfsmred}
\setbeamercolor*{enumerate subsubitem}{fg=utfsmred}

\setbeamercolor*{itemize item}{fg=utfsmred}
\setbeamercolor*{itemize subitem}{fg=utfsmred}
\setbeamercolor*{itemize subsubitem}{fg=utfsmred}

\setbeamercolor{item projected}{bg=utfsmred}


\setbeamertemplate{itemize items}[square]
\setbeamertemplate{enumerate items}[default]


\setbeamercolor{section in head/foot}{bg=utfsmblue}

\setbeamercolor{block title}{bg=utfsmblue!80,fg=white}
\setbeamercolor{block title alerted}{bg=utfsmred!80,fg=white}
\setbeamercolor{block title example}{bg=utfsmgreen!80,fg=white}

\setbeamertemplate{sections/subsections in toc}[square]
\setbeamercolor{section number projected}{bg=utfsmblue,fg=white}


}


\usepackage{graphicx} % Allows including images
\usepackage{booktabs} % Allows the use of \toprule, \midrule and \bottomrule in tables
\usepackage{listings}
\lstset{ %
basicstyle=\normalsize\ttfamily,
keywordstyle=,
numbers=none,
numberstyle=\tiny\ttfamily,
stepnumber=1,
showspaces=false,
showstringspaces=false,
showtabs=false,
breaklines=true,
frame=tb,
framerule=0.5pt,
tabsize=4,
framexleftmargin=0.5em,
framexrightmargin=0.5em,
xleftmargin=0.5em,
xrightmargin=0.5em,
}

%----------------------------------------------------------------------------------------
%	TITLE PAGE
%----------------------------------------------------------------------------------------

\title{Linux Básico}
\subtitle{\small{Seminario de Desarrollo de Software - Casa Central.}}
\author{Maximiliano Osorio\\\small{mosorio@inf.utfsm.cl}} 
\institute[UTFSM]
{
Universidad Técnica Federico Santa María
\medskip
}
\date{\today} % Date, can be changed to a custom date

\begin{document}

\maketitle

\begin{frame}[fragile]
\frametitle{Bash shell}
\begin{itemize}
	\item \textit{Command line} es una interfaz de texto para recibir instrucciones. En Linux el programa que lo provee es llamado shell.
	\item Por defecto la shell es RedHat es llamada Bash, existen otras como: sh, zsh, etc.
	\item Prompt: Es el caracter cuando está esperando.
	\item El \textbf{command} es el nombre del programa a correr.
	\item Los \textbf{commands} se dividen:
	\begin{itemize}
		\item Command a correr
		\item Options, para ajustar el comportamiento del comando.
		\item Arguments
	\end{itemize}
	
	\begin{lstlisting}
		$ usermod -L morgan
	\end{lstlisting}
\end{itemize}
	
\end{frame}
\section{Ayuda}
\subsection{Manuales}
\begin{frame}[fragile]
\frametitle{¿Cómo pedir ayuda?}


\begin{itemize}
	\item Los \textbf{commands} tienen su ayuda en el programa.
	\begin{itemize}
		\item date --help
	\end{itemize}
	\begin{lstlisting}
date --help
Usage: date [OPTION]... [+FORMAT]
  or:  date [-u|--utc|--universal] [MMDDhhmm[[CC]YY][.ss]]
Display the current time in the given FORMAT, or set the system date.
**** output truncado ****
	\end{lstlisting}
\end{itemize}
 
\end{frame}

\begin{frame}[fragile]
	\frametitle{man}
	\textbf{man} es el comando para acceder a los manuales de Linux.
	\begin{itemize}
		\item Los manuales están construidos en la mayoria de los sistemas Linux.
		\item Proveen una documentación extensa sobre los \textit{commands} y otros aspectos del sistemas.
	\end{itemize}
	\begin{lstlisting}
		$ man -k date
	\end{lstlisting}
\end{frame}

\begin{frame}[fragile]
\frametitle{man}
\begin{itemize}
	\item Brackets [] representa items opcionales.
	\item Cualquier cosa seguida de tres puntos (...) representa una lista de items del largo arbitario
	\item Multiples items separados por pipes | representa que solo una opción puede ser especificada.
	\item Texto en brackets \verb|<>| valor mandatorio.
\end{itemize}

\begin{lstlisting}
NAME
       date - print or set the system date and time

SYNOPSIS
       date [OPTION]... [+FORMAT]
       date [-u|--utc|--universal] [MMDDhhmm[[CC]YY][.ss]]	
\end{lstlisting}
\end{frame}



\begin{frame}[fragile]
\frametitle{Ejemplos - date}

\begin{lstlisting}
[mosorio@ssh ~]$ date
Mon Oct  5 21:54:57 CLT 2015
\end{lstlisting}
\begin{lstlisting}
[mosorio@ssh ~]$ date +%R
21:55
\end{lstlisting}
\begin{lstlisting}
[mosorio@ssh ~]$ date +%x
10/05/2015
\end{lstlisting}

\begin{lstlisting}
[mosorio@ssh ~]$ passwd
\end{lstlisting}
\end{frame}

\section{Ejemplos de comandos}
\begin{frame}[fragile]
\frametitle{Ejemplos - passwd}
\begin{lstlisting}
[mosorio@losvilos ~]$ passwd
Changing password for user mosorio.
Changing password for mosorio.
(current) UNIX password:
New password:
Retype new password:
passwd: all authentication tokens updated successfully.
\end{lstlisting}
\end{frame}


\begin{frame}[fragile]
\frametitle{Tipo de archivos}
Linux no necesita extensiones para identificar los archivos. \textbf{file} hace un scan para detectar el tipo.

\begin{lstlisting}
[root@losvilos ~]# file /bin/passwd
/bin/passwd: setuid ELF 64-bit LSB shared object, x86-64, version 1 (SYSV), dynamically linked, interpreter /lib64/ld-linux-x86-64.so.2, for GNU/Linux 2.6.32, BuildID[sha1]=4984e8b5a58b4a2ed119be78eff90ca79fd0ebb6, stripped
\end{lstlisting}
\end{frame}

\begin{frame}[fragile]
\frametitle{cat	}
cat es una buena 
\begin{lstlisting}
[mosorio@ssh ~]$ cat /etc/passwd
root:x:0:0:root:/root:/bin/bash
bin:x:1:1:bin:/bin:/sbin/nologin
daemon:x:2:2:daemon:/sbin:/sbin/nologin
adm:x:3:4:adm:/var/adm:/sbin/nologin
lp:x:4:7:lp:/var/spool/lpd:/sbin/nologin
sync:x:5:0:sync:/sbin:/bin/sync
shutdown:x:6:0:shutdown:/sbin:/sbin/shutdown
halt:x:7:0:halt:/sbin:/sbin/halt
mail:x:8:12:mail:/var/spool/mail:/sbin/nologin

...
\end{lstlisting}

\end{frame}


\begin{frame}[fragile]
\frametitle{head}
\begin{lstlisting}
[root@losvilos ~]# head -n 5 /etc/passwd
root:x:0:0:root:/root:/bin/bash
bin:x:1:1:bin:/bin:/sbin/nologin
daemon:x:2:2:daemon:/sbin:/sbin/nologin
adm:x:3:4:adm:/var/adm:/sbin/nologin
lp:x:4:7:lp:/var/spool/lpd:/sbin/nologin
\end{lstlisting}
\end{frame}

\begin{frame}[fragile]
\frametitle{COUNT}
\begin{lstlisting}
[root@losvilos ~]# wc /etc/passwd
  64  148 3531 /etc/passwd
[root@losvilos ~]# wc -l /etc/passwd
64 /etc/passwd
\end{lstlisting}
\end{frame}


\begin{frame}[fragile]
\frametitle{tab tab}
\begin{lstlisting}
[root@losvilos ~]# pas<tab>
passmass     passwd       paste        pasuspender
[root@losvilos ~]# pass<tab>
passmass  passwd
[root@losvilos ~]# passwd
\end{lstlisting}

\begin{lstlisting}
[root@losvilos ~]# useradd --<tab>	
\end{lstlisting}
\end{frame}


\begin{frame}[fragile]
\frametitle{History}
\begin{lstlisting}
[root@losvilos ~]# history
 2619  Oct/05 - 22:03:27 su - mosorio
 2620  Oct/05 - 22:14:13 file /etc/passwd
 2621  Oct/05 - 22:14:48 file /etc/shadow
 2622  Oct/05 - 22:14:59 file /bin/passwd
 2623  Oct/05 - 22:24:22 head -n 5 /etc/passwd
 2624  Oct/05 - 22:24:45 wc /etc/passwd
 2625  Oct/05 - 22:24:51 wc -l /etc/passwd
 2626  Oct/05 - 22:30:58 history	
\end{lstlisting}

\begin{lstlisting}
[root@losvilos ~]# !wc
wc -l /etc/passwd
64 /etc/passwd
\end{lstlisting}
\end{frame}

\begin{frame}{Atajos}
\begin{table}[]
\centering
\label{my-label}
\begin{tabular}{|l|l|}
\hline
Comando          & Times {[}s{]}                      \\ \hline
ctrl+a           & Inicio de la linea de comandos     \\ \hline
ctrl+e           & Final de la linea de comandos      \\ \hline
ctrl+u           & Limpiar desde el inicio al cursor  \\ \hline
ctrl+k           & Limpiar desde el cursor al final   \\ \hline
ctrl+left arrow  & Salta a la palabra izquierda       \\ \hline
ctrl+right arrow & Salta a la palabra derecha         \\ \hline
ctrl+r           & Buscar en el history por un patrón \\ \hline
ctrl+d           & Salir de la shell \\ \hline

\end{tabular}
\end{table}
\end{frame}


\begin{frame}{Ejercicio}
	\begin{enumerate}
		\item Muestre los accesos que están ocurriendo en su servidor web. \texttt{/var/log/nginx/access.log}
		\item Muestre los últimos 10 errores en su servidor en nginx  \texttt{/var/log/nginx/error.log}
		\item Muestre la configuración de red de la interfaz eth0 de su servidor, se encuentra en \texttt{/etc/sysconfig/network-scripts/ifcfg-eth0}
		\item Tipo de archivo del archivo /etc/passwd ¿es leible por humanos?
		\item ext4 es un sistema de archivo, busque la documentación para este tipo de archivo.
		\item Utilice ctrl+r para mostrar la hora como en el paso 1.
	\end{enumerate}
\end{frame}

\end{document}