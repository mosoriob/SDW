
\documentclass[xcolor=dvipsnames]{beamer} % dvipsnames gives more built-in colors
\usepackage[utf8]{inputenc}
\usepackage[spanish]{babel}

\mode<presentation> {

\usetheme{CambridgeUS}

%\setbeamertemplate{footline} % To remove the footer line in all slides uncomment this line
\setbeamertemplate{footline}[page number] % To replace the footer line in all slides with a simple slide count uncomment this line

\setbeamertemplate{navigation symbols}{} % To remove the navigation symbols from

\definecolor{utfsmred}{HTML}{D60019}
\definecolor{utfsmyellow}{HTML}{F7AE00}
\definecolor{utfsmgreen}{HTML}{008452}
\definecolor{utfsmblue}{HTML}{004B85}


\newenvironment<>{rosa}[1][]{
  \setbeamercolor{block title example}{fg=white,bg=blue!75!black}%
  \begin{example}#2[#1]}{
  \end{example}
}



\usecolortheme[named=utfsmblue]{structure}
\setbeamercolor{titlelike}{parent=structure,fg=utfsmblue}
\setbeamercolor{frametitle}{fg=utfsmblue}

%\setbeamercolor{section in head/foot}{bg=Brown}
%\setbeamercolor{author in head/foot}{bg=Brown}
%\setbeamercolor{date in head/foot}{fg=Brown}

\setbeamercolor*{enumerate item}{fg=utfsmred}
\setbeamercolor*{enumerate subitem}{fg=utfsmred}
\setbeamercolor*{enumerate subsubitem}{fg=utfsmred}

\setbeamercolor*{itemize item}{fg=utfsmred}
\setbeamercolor*{itemize subitem}{fg=utfsmred}
\setbeamercolor*{itemize subsubitem}{fg=utfsmred}

\setbeamercolor{item projected}{bg=utfsmred}


\setbeamertemplate{itemize items}[square]
\setbeamertemplate{enumerate items}[default]


\setbeamercolor{section in head/foot}{bg=utfsmblue}

\setbeamercolor{block title}{bg=utfsmblue!80,fg=white}
\setbeamercolor{block title alerted}{bg=utfsmred!80,fg=white}
\setbeamercolor{block title example}{bg=utfsmgreen!80,fg=white}

\setbeamertemplate{sections/subsections in toc}[square]
\setbeamercolor{section number projected}{bg=utfsmblue,fg=white}


}


\usepackage{graphicx} % Allows including images
\usepackage{booktabs} % Allows the use of \toprule, \midrule and \bottomrule in tables
\usepackage{listings}
\lstset{ %
language=C,
basicstyle=\normalsize\ttfamily,
keywordstyle=,
numbers=none,
numberstyle=\tiny\ttfamily,
stepnumber=1,
showspaces=false,
showstringspaces=false,
showtabs=false,
breaklines=true,
frame=tb,
framerule=0.5pt,
tabsize=4,
framexleftmargin=0.5em,
framexrightmargin=0.5em,
xleftmargin=0.5em,
xrightmargin=0.5em,
}

%----------------------------------------------------------------------------------------
%	TITLE PAGE
%----------------------------------------------------------------------------------------

\title{Systemd}
\subtitle{\small{Seminario de Desarrollo de Software - Casa Central.}}
\author{Maximiliano Osorio\\\small{mosorio@inf.utfsm.cl}} 
\institute[UTFSM]
{
Universidad Técnica Federico Santa María
\medskip
}
\date{\today} % Date, can be changed to a custom date

\begin{document}
	
%-=-=-=-=-=-=-=-=-=-=-=-=-=-=-=-=-=-=-=-=-=-=-=-=
%
%	TITLE PAGE
%
%-=-=-=-=-=-=-=-=-=-=-=-=-=-=-=-=-=-=-=-=-=-=-=-=




\maketitle

\begin{frame}[fragile]
\frametitle{Daemon}

Daemons son procesos que esperan o corren en background realizando diversas tareas.

Generalmente daemons corren de manera automática cuando se inicia el sistema (boot) y continuan hasta al shutdown o cuando son parados de manera manual.
\end{frame}
%------------------------------------------------


\begin{frame}[fragile]
    \frametitle{init}
Por durante muchos años, process ID 1 de Linux fue llamado init process. Este era el responsable de activar los otros servicios del sistema. 
Normalmente los daemons eran iniciados por System V y LSB scripts. Estos son scripts y variaban de distro a distro.

En RHEL 7, proceso ID1 es systemd, el nuevo sistema de init.

\end{frame}

%------------------------------------------------

\begin{frame}[fragile]
\frametitle{Init}
\begin{itemize}
  \item threadd process is responsible for creating additional processes that perform tasks on behalf of the kernel
\end{itemize}

\end{frame}
%------------------------------------------------


\begin{frame}[fragile]
    \frametitle{System}
System startup y ciertos procesos son manejados por System and Service Manager (systemd). \\
Este programa provee una forma de activar recursos del sistema, server daemons y otros procesos.
\end{frame}


\begin{frame}[fragile]
\frametitle{Ejemplo}
nginx
\end{frame}
%------------------------------------------------




\begin{frame}[fragile]
    \frametitle{Variables}

Las variables se encuentran en \textbf{/etc/sysconfig}. Y en los archivos se pueden encontrar definiciones en pares de NAME=VALUE.

\begin{lstlisting}
cat /etc/sysconfig/docker
\end{lstlisting}
\end{frame}


\begin{frame}[fragile]
    \frametitle{systemctl}
	\textbf{systemctl} es usado para manejar diversos tipos de objetos, llamado \textbf{units}
	Se pueden obtener con:
	\begin{lstlisting}
		systemctl -l
	\end{lstlisting}
	Tipos de \textbf{units}
	\begin{itemize}
		\item Socket units termina en .socket
		\item Service units terminan en .service. Este tipo de unit es usado para trabajar con daemons.
	\end{itemize}
\end{frame}



\begin{frame}[fragile]
    \frametitle{Service states}
	\textbf{systemctl status name.type} muestra el estado de una unidad.  Si el tipo de la unidad no es especificada, se muestra el name.service.
	Se pueden obtener con:

\end{frame}

\begin{frame}[fragile]
    \frametitle{Service states}
	\begin{lstlisting}[basicstyle=\small]
systemctl status sshd.service
sshd.service - OpenSSH server daemon
   Loaded: loaded (/usr/lib/systemd/system/sshd.service; enabled)
   Active: active (running) since Sun 2015-11-22 12:34:51 ACT; 10h ago
 Main PID: 1206 (sshd)
   CGroup: /system.slice/sshd.service
           1206 /usr/sbin/sshd -D

Nov 22 12:34:51 ip122.15.priv.inf.utfsm.cl sshd[1206]: Server listening on 0.0.0.0 port 22.
Nov 22 12:34:51 ip122.15.priv.inf.utfsm.cl sshd[1206]: Server listening on :: port 22.
	\end{lstlisting}

\end{frame}

\begin{frame}{Estados}
\begin{itemize}
	\item loaded: El archivo de configuración ha sido procesado.
	\item active (running): Corriendo con uno o mas procesos.
	\item active (exited): Exitosamente completado.
	\item active (waiting): Corriendo pero esperando un evento.
	\item inactive: No corriendo
	\item enabled: Va a correr en booteo
	\item disabled: No va a correr en el booteo
\end{itemize}
\end{frame}

\begin{frame}[fragile]
\frametitle{Controlando los servicios}
Viendo el estado
\begin{lstlisting}
	systemctl status sshd.service	\end{lstlisting}

Iniciar el servicio
\begin{lstlisting}
	systemctl start sshd.service
\end{lstlisting}
Parar el servicio
\begin{lstlisting}
	systemctl stop sshd.service
\end{lstlisting}
Reload de la configuración, el PID no va a cambiar (HUP).
\begin{lstlisting}
	systemctl reload sshd.service
\end{lstlisting}
\end{frame}

\begin{frame}[fragile]
\frametitle{Unit dependencies}
Puede ser utilizado para mostrar un árbol de dependencias.
\begin{lstlisting}
systemctl list-dependencies docker
\end{lstlisting}	

\end{frame}



\end{document}